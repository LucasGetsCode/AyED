\documentclass[10pt,a4paper]{article}

\input{AEDmacros}
\usepackage{caratula} % Version modificada para usar las macros de algo1 de ~> https://github.com/bcardiff/dc-tex


\titulo{Trabajo Práctico 1: Especificación y WP}
\subtitulo{Elecciones Nacionales}

\fecha{\today}

\materia{Algoritmos y Estructuras de Datos}
\grupo{Grupo.java}

\integrante{Apellido, Nombre1}{001/01}{email1@dominio.com}
\integrante{Apellido, Nombre2}{002/01}{email2@dominio.com}
\integrante{Apellido, Nombre3}{003/01}{email3@dominio.com}
\integrante{Apellido, Nombre4}{004/01}{email4@dominio.com}
% Pongan cuantos integrantes quieran

% Declaramos donde van a estar las figuras
% No es obligatorio, pero suele ser comodo
\graphicspath{{../static/}}

\begin{document}

\maketitle

\section{Especificación}
\begin{enumerate}
    \item \textbf{hayBallotage}: verifica si hay ballotage en la elección presidencial.
    \begin{proc}{hayBallotage}{\In escrutinio: \TLista{\ent}}{\bool}
	%    \modifica{parametro1, parametro2,..}
	\requiere{condicionesBasicas(escrutinio, 1)}
	\asegura{\res = \neg (tieneMas45\%(escrutinio) \lor tieneMas40\%Diferencia10\%(escrutinio))}
 
    \pred{tieneMas45\%}{\In escrutinio: \TLista{\ent}} {\existe[unalinea]{i}{\ent}{0 \leq i< \longitud{escrutinio}-1 \yLuego escrutinio[i]\neq 0 \yLuego porcentaje(escrutinio, i) > 45}}
    
    \pred{tieneMas40\%Diferencia10\%}{\In escrutinio: \TLista{\ent}} {\existe[unalinea]{i}{\ent}{0 \leq i< \longitud{escrutinio}-1\yLuego escrutinio[i]\neq 0 \yLuego (porcentaje(escrutinio, i) > 40 \land \\ \existe[unalinea]{j}{\ent}{(0 \leq j < \longitud{escrutinio} - 1 \yLuego escrutinio[j] \neq 0 \land j \neq i) \yLuego \\ porcentaje(escrutinio, i) - porcentaje(escrutinio, j) > 10})}}
\end{proc}

    \item \textbf{hayFraude}: verifica que los votos válidos de los tres tipos de cargos electivos sumen lo mismo.
    \begin{proc}{hayFraude}{\In escrutinio\_presidencial: \TLista{\ent}, \In escrutinio\_senadores: \TLista{\ent}, \In escrutinio\_diputados: \TLista{\ent}}{\bool}
	%    \modifica{parametro1, parametro2,..}
	\requiere{condicionesBasicas(escrutinio\_presidencial,1)\yLuego \\condicionesBasicas(escrutinio\_senadores,2) \yLuego \\ condicionesBasicas(escrutinio\_diputados,1)}
	\asegura{expresionBooleana2}
\end{proc}

    \item \textbf{obtenerSenadoresEnProvincia}: obtiene los id de los partidos (primero y segundo) para la elección de senadores en una provincia. El id es el índice de las listas escrutinios.
    \begin{proc}{obtenerSenadoresEnProvincia}{\In escrutinio: \TLista{\ent}}{$\ent \times \ent$}
	%    \modifica{parametro1, parametro2,..}
	\requiere{expresionBooleana1}
	\asegura{expresionBooleana2}
\end{proc}
 
    \item \textbf{calcularDHondtEnProvincia}: calcula los cocientes según el método d’Hondt para diputados en una provincia (importante: no es necesario ordenar los partidos por cantidad de votos)
    \begin{proc}{calcularDHondtEnProvincia}{\In cant\_bancas: \ent, \In escrutinio: \TLista{\ent}}{\TLista{\TLista{\ent}}}
	%    \modifica{parametro1, parametro2,..}
	\requiere{condicionesBasicas(escrutinio, 1) \yLuego superanUmbralElectoral(escrutinio, 1) \land cant\_bancas > 0}
	\asegura{\longitud{res}=\longitud{escrutinio}-1 \yLuego \paraTodo{i, j}{\ent}{0 \leq i < \longitud{res} \implicaLuego \longitud{res[i]}=cant\_bancas \yLuego \\ 0 \leq j < \longitud{res[i]} \implicaLuego res[i][j] = \lfloor \frac{escrutinio[i]}{1+j} \rfloor}}
\end{proc}

    \item \textbf{obtenerDiputadosEnProvincia}: calcula la cantidad de bancas de diputados obtenidas por cada partido en una provincia
    \begin{proc}{obtenerDiputadosEnProvincia}{\In cant\_bancas: \ent, \In dHondt: \TLista{\TLista{\ent}}}{\TLista{\ent}}
	%    \modifica{parametro1, parametro2,..}
	\requiere{esMatriz(dHondt) \yLuego (estaOrdenado(dHondt) \land cocientesPositivos(dHondt) \land \\ noExistenCocientesRepetidos(dHondt) \land 0 < cant\_bancas \leq \longitud{dHondt[0]} )}
	\asegura{\longitud{res} = \longitud{dHondt} \land \paraTodo[unalinea]{i}{\ent}{0 \leq i < \longitud{res} \implicaLuego 0 \leq res[i] \leq cant\_bancas} \land \\ \displaystyle\sum_{i=0}^{\longitud{res}-1} res[i] = cant\_bancas \land distribucionBancasCocientes(cant\_bancas, dHondt)}
 
    \pred{esDHondtValido}{\In dH: \TLista{\TLista{\ent}}}{esMatriz(dH) \yLuego (estaOrdenado(dH) \land cocientesPositivos(dH))}

    \pred{esMatriz}{\In matriz: \TLista{\TLista{\ent}}}{\longitud{matriz} > 0 \yLuego \longitud{matriz[0]} > 0 \yLuego (\paraTodo[unalinea]{l}{\ent}{0 < l < \longitud{matriz} \implicaLuego \longitud{matriz[l]} = \longitud{matriz[l-1]}} }

    \pred{estaOrdenado}{\In dH: \TLista{\TLista{\ent}}}{\paraTodo[unalinea]{i, j}{\ent}{0 \leq i < \longitud{dH} \land 0 \leq j < \longitud{dH[0]} \implicaLuego \\ (dH[i][0] \geq dH[i][j] * (j+1) \land dH[i][0] < (dH[i][j]+1) * (j+1)))}}

    \pred{cocientesPositivos}{\In dH: \TLista{\TLista{\ent}}}{\paraTodo[unalinea]{i, j}{\ent}{0 \leq i < \longitud{dH} \land 0 \leq j < \longitud{dH[0]} \implicaLuego dH[i][j] \geq 0}}
    
    \pred{noExistenCocientesRepetidos}{\In dH: \TLista{\TLista{\ent}}}{\paraTodo[unalinea]{i,j,i',j'}{\ent}{0\leq i,i' < \longitud{dH} \land 0 \leq j,j' < \longitud{dH[0]} \land (i = i' \implica j \neq j') \implicaLuego dH[i][j] \neq dH[i'][j']}}
    
    \pred{distribucionBancasCocientes}{\In cant\_bancas: \ent, \In dH: \TLista{\TLista{\ent}}}{\paraTodo[unalinea]{i}{\ent}{0\leq i < \longitud{res} \implicaLuego (\displaystyle\sum_{part=0}^{\longitud{dH}-1} (\displaystyle\sum_{coc=0}^{\longitud{dH[0]}-1} \IfThenElse{dH[part][coc] > dH[i][res[i]]}{1}{0})) \geq cant\_bancas}}
\end{proc}

    \item \textbf{validarListasDiputadosEnProvincia}: verifica que la listas de diputados de cada partido en una provincia contenga exactamente la misma cantidad de candidatos que bancas en disputa en esa provincia, y que además se cumpla la alternancia de géneros.
    \begin{proc}{validarListasDiputadosEnProvincia}{\In cant\_bancas: \ent, \In listas: \TLista{\TLista{$dni : $\ent\ \times$ genero : \ent$}}}{\bool}
	%    \modifica{parametro1, parametro2,..}
	\requiere{expresionBooleana1}
	\asegura{expresionBooleana2}
\end{proc}

    \item \textbf{Auxiliares y predicados sueltos}\\[1ex]
    \pred{condicionesBasicas}{\In escrutinio: \TLista{\ent}, \In partidos : \ent}
    {hayPartidosYblancos(escrutinio, partidos)\yLuego (\neg hayVotosNegativos(escrutinio)\land\neg hayEmpate(escrutinio))}
    \pred{hayPartidosYblancos}{\In escrutinio: \TLista{\ent}, \In partidos : \ent} {\longitud{escrutinio}\geq partidos+1}
    \pred{hayVotosNegativos}{\In escrutinio: \TLista{\ent}} {\existe[unalinea]{i}{\ent}{0 \leq i< \longitud{escrutinio} \yLuego escrutinio[i]<0}}
    \pred{hayEmpates}{\In escrutinio: \TLista{\ent}} {\existe[unalinea]{i,j}{\ent}{(0 \leq i,j< \longitud{escrutinio}-1 \land i\neq j) \implicaLuego escrutinio[i]=escrutinio[j]}}
    \pred{superanUmbralElectoral}{\In escrutinio: \TLista{\ent}, \In cantidad: \ent} {(\displaystyle\sum_{i=0}^{\longitud{lista}-2} \IfThenElse{porcentaje(i, escrutinio) > 3}{1}{0}) \geq cantidad}
    
    \aux{porcentaje}{\In lista: \TLista{\ent}, \In indice: \ent}{ \float }{ 100*\frac{\displaystyle\sum_{i=0}^{\longitud{lista}-1} lista[i]}{lista[indice]} }    
    
\end{enumerate}


\section{Implementación y demostraciones de correctitud}

\end{document}